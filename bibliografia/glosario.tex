%%%%%%%%%%%%%%%%%%%%%%%%%%%%%%%%%%%%%%%%%%%%%%%%%%%%%%%%%%%%%%%%%%%%%%%%%%%%%%%%
% Obxectivo: Lista de termos empregados no documento,                          %
%            xunto cos seus respectivos significados.                          %
%%%%%%%%%%%%%%%%%%%%%%%%%%%%%%%%%%%%%%%%%%%%%%%%%%%%%%%%%%%%%%%%%%%%%%%%%%%%%%%%
\newglossaryentry{DFV}{
  name=DFV,
  description={Siglas en inglés de \textit{Deformation Vector Field}.
  Consiste nunha matriz que contén, para cada punto, un vector que indica a magnitude e dirección do desprazamento a realizar.}
}

\newglossaryentry{INR}{
  name=INR,
  description={Siglas en inglés de \textit{Implicit Neural Representations}.
  Son representacións de funcións continuas parametrizadas por redes neuronais artificiais que modelan obxectos ou sinais, como imaxes ou formas 3D.
  Habitualmente mapean coordenadas de entrada directamente a valores de saída relevantes, en lugar de almacenar datos de forma discreta.}
}

\newglossaryentry{MLP}{
  name=MLP,
  description={Siglas en inglés de \textit{Multi-Layer Perceptrón}.
  Tipo de red neuronal artificial formada por capas completamente conectadas de neuronas, que incluye una capa de entrada, una o más capas ocultas, y una capa de salida.
  }
}
\newglossaryentry{CNN}{
  name=CNN,
  description={Siglas en inglés de \textit{Convolutional Neural Network}.
  Tipo de rede neuronal artificial que utiliza capas convolucionales que aplican filtros para extraer diferentes características, permitindo o recoñecemento de patrones complexos.
  }
}

\newglossaryentry{GAN}{
  name=GAN,
  description={Siglas en inglés de \textit{Generative Adversarial Network}.
  Tipo de rede neuronal artificial composta por dúas redes que compiten entre si: un xerador, que intenta crear datos falsos realistas, e un discriminador, que tenta distinguir entre datos reais e falsos.
  }
}

\newglossaryentry{IGRT}{
  name=IGRT,
  description={Siglas en inglés de \textit{Image-Guided Radiation Therapy}.
  Técnica de radioterapia que utiliza imaxes médicas obtidas inmediatamente antes ou durante o tratamento para localizar con precisión o tumor e adaptar a administración da radiación, mellorando a exactitude e minimizando o dano aos tecidos sans circundantes.
  }
}

\newglossaryentry{RIR}{
  name=RIR,
  description={Siglas en inglés de \textit{Retinal Image Registration}.
  Técnica utilizada para aliñar ou superpoñer imaxes da retina obtidas en diferentes momentos, condicións ou dispositivos, co obxectivo de comparar cambios, monitorizar enfermidades ou mellorar a análise clínica.
  }
}


% \newglossaryentry{PSO}{
%   name=PSO,
%   description={Siglas en inglés de \textit{Particle Swarm Optimization}.
%   Algoritmo de optimización inspirado no comportamento colectivo de bandadas de paxaros ou bancos de peixes. Utiliza un conxunto de partículas (solucións candidatas) que exploran o espazo de busca movéndose segundo a súa experiencia propia e a dos seus veciños, co obxectivo de atopar a mellor solución posible a un problema dado.
%   }
% }

% \newglossaryentry{REMPE}{
%   name=REMPE,
%   description={Siglas en inglés de \textit{Registration of Retinal Images Through Eye Modelling and Pose Estimation}.
%   }
% }

\newglossaryentry{PIIFD}{
  name=PIIFD,
  description={Siglas de \textit{Partial Intensity Invariant Feature Descriptor}.
  É un descriptor de características que xera representación robustas de contornos en imaxes, mantendo certa invariancia frente a cambios parciales de intensidade e transformacións xeométricas.}
}


\newglossaryentry{BBF}{
  name=BBF,
  description={Siglas en inglés de \textit{Best Bin First}. Algoritmo de busca de veciños en espazos de alta dimensión, empregado na comparación de descritores.}
}


\newglossaryentry{SIFT}{
  name=SIFT,
  description={Siglas en inglés de \textit{Scale-Invariant Feature Transform}. 
  Algoritmo para detectar e describir puntos de interese en imaxes, robusto a cambios de escala, rotación e iluminación.}
}

\newglossaryentry{SURF}{
  name=SURF,
  description={Siglas en inglés de \textit{Speeded-Up Robust Features}. 
  Algoritmo para a detección e descrición de características locais en imaxes, máis rápido que SIFT, pero pode ser algo menos preciso.}
}
% \newglossaryentry{UR-SIFT-PIIFD}{
%   name=UR-SIFT-PIIFD,
%   description={Siglas en inglés de \textit{Uniform Robust Scale Invariant Feature Transform e Harris-Partial Intensity Invariant Feature Descriptor}.
%   }
% }


\newglossaryentry{GMM}{
  name=GMM,
  description={Siglas en inglés de \textit{Gaussian Mixture Models}. 
  Modelo estatístico que representa a distribución de datos como unha combinación de varias gaussianas. Funciona axustando os parámetros das gaussianas para modelar a estrutura dos datos e identificar subconxuntos ou clusters.}
}

\newglossaryentry{GDB-ICP}{
  name=GDB-ICP,
  description={Siglas en inglés de \textit{Generalized Dual-Bootstrap Iterative Closest Point}.
   Algoritmo de aliñamento iterativo de puntos. Funciona buscando correspondencias entre puntos e refinando a transformación ata conseguir o mellor axuste posible.}
}

\newglossaryentry{FOV}{
  name=FOV,
  description={Siglas en inglés de \textit{Field Of View}.
   Ángulo ou área visible a través dun sensor ou dispositivo óptico.Determina a rexión do espazo que pode captar ou observar un sistema de visión.}
}


\newglossaryentry{NTK}{
  name=NTK,
  description={Siglas en inglés de \textit{Neural Tangent Kernel}. Ferramenta matemática para analizar o comportamento de redes neuronais durante o adestramento.
   Funciona aproximando a evolución dos parámetros da rede mediante un kernel que describe a súa dinámica de aprendizaxe.
   }
}


\newglossaryentry{4DCT}{
  name=4DCT,
  description={Siglas en inglés de \textit{Four-Dimensional Computed Tomography}. La Tomografía Computarizada del tórax en cuatro dimensiones
  es una técnica de imagen médica que permite captura información dinámica del tórax, enfocándose principalmente en los pulmones.
  La cuarta dimensión se refiere al tiempo, lo que permite capturar el movimiento de las estructuras a lo largo del ciclo respiratorio, 
  generando así una serie de imágenes 3D sincronizadas con las fases de la respiración.
  }
}

\newglossaryentry{IBR}{
  name=IBR,
  description={Siglas en inglés de \textit{Image-Based Registration}.
  Técnica de rexistro de imaxes baseada na comparación directa dos valores de intensidade dos píxeles ou voxeles das imaxes a aliñar.
  }
}

\newglossaryentry{FBR}{
  name=FBR,
  description={Siglas en inglés de \textit{Feature-Based Registration}.
  Técnica de rexistro de imaxes baseada na identificación e correspondencia de características salientables, como puntos, liñas ou bordes, presentes nas imaxes.
  }
}

