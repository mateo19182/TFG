%%%%%%%%%%%%%%%%%%%%%%%%%%%%%%%%%%%%%%%%%%%%%%%%%%%%%%%%%%%%%%%%%%%%%%%%%%%%%%%%
% Obxectivo: Lista de termos empregados no documento,                          %
%            xunto cos seus respectivos significados.                          %
%%%%%%%%%%%%%%%%%%%%%%%%%%%%%%%%%%%%%%%%%%%%%%%%%%%%%%%%%%%%%%%%%%%%%%%%%%%%%%%%

\newglossaryentry{DFV}{
  name=DFV,
  description={Siglas en inglés de \textit{Campo de Vectores de Deformación}.}
}


\newglossaryentry{INR}{
  name=DFV,
  description={Siglas en inglés de \textit{Redes Neuronais Implícitas}.}
}

\newglossaryentry{MLP}{
  name=MLP,
  description={Siglas en inglés de \textit{Multi-Layer Perceptrón}.
  Tipo de red neuronal artificial formada por capas completamente conectadas de neuronas, que incluye una capa de entrada, una o más capas ocultas, y una capa de salida.
  }
}

\newglossaryentry{CNN}{
  name=CNN,
  description={Siglas en inglés de \textit{Convolutional Neural Network}.
  Tipo de red .
  }
}

\newglossaryentry{GAN}{
  name=CNN,
  description={Siglas en inglés de \textit{Generative Adversarial Network}.
  Tipo de red .
  }
}

\newglossaryentry{IGRT}{
  name=IGRT,
  description={Siglas en inglés de \textit{Image-guided radiation therapy}.
  }
}


\newglossaryentry{RIR}{
  name=RIR,
  description={Siglas en inglés de \textit{Retinal Image Registration}.
  }
}


\newglossaryentry{PSO}{
  name=PSO,
  description={Siglas en inglés de \textit{Particle Swarm Optimization}.
  }
}

\newglossaryentry{REMPE}{
  name=REMPE,
  description={Siglas en inglés de \textit{Registration of Retinal Images Through Eye Modelling and Pose Estimation}.
  }
}

\newglossaryentry{PIIFD}{
  name=PIIFD,
  description={Siglas en inglés de \textit{Partial Intensity Invariant Feature Descriptor}.
  }
}

\newglossaryentry{BBF}{
  name=PIIFD,
  description={Siglas en inglés de \textit{Best Bin First}.
  }
}

\newglossaryentry{SIFT}{
  name=SIFT,
  description={Siglas en inglés de \textit{Scale-Invariant Feature Transform}.
  }
}

\newglossaryentry{SURF}{
  name=SURF,
  description={Siglas en inglés de \textit{Speeded-Up Robust Features}.
  }
}

\newglossaryentry{UR-SIFT-PIIFD}{
  name=UR-SIFT-PIIFD,
  description={Siglas en inglés de \textit{Uniform Robust Scale Invariant Feature Transform e Harris-Partial Intensity Invariant Feature Descriptor}.
  }
}

\newglossaryentry{MFSP}{
  name=MFSP,
  description={Siglas en inglés de \textit{Mixture Feature and Structural Preservation Feature Points Registration}.
  }
}

\newglossaryentry{GMM}{
  name=GMM,
  description={Siglas en inglés de \textit{Gaussian Mixture Models}.
  }
}

\newglossaryentry{TPS}{
  name=TPS,
  description={Siglas en inglés de \textit{Thin Plated Spine}.
  }
}

\newglossaryentry{GDB-ICP}{
  name=GDB-ICP,
  description={Siglas en inglés de \textit{Generalized Dual-Bootstrap Iterative Closest Point}.
  }
}

\newglossaryentry{FOV}{
  name=FOV,
  description={Siglas en inglés de \textit{Field Of View}.
  }
}

\newglossaryentry{NTK}{
  name=NTK,
  description={Siglas en inglés de \textit{Neural Tangent Kernel}.
  }
}


\newglossaryentry{4DCT}{
  name=4DCT,
  description={Siglas en inglés de \textit{Four-Dimensional Computed Tomography}. La Tomografía Computarizada del tórax en cuatro dimensiones
  es una técnica de imagen médica que permite captura información dinámica del tórax, enfocándose principalmente en los pulmones.
  La cuarta dimensión se refiere al tiempo, lo que permite capturar el movimiento de las estructuras a lo largo del ciclo respiratorio, 
  generando así una serie de imágenes 3D sincronizadas con las fases de la respiración.
  }
}

\newglossaryentry{IBR}{
  name=IBR,
  description={Siglas en inglés de \textit{Image-Based Registration}.
  Técnica de rexistro de imaxes baseada na comparación directa dos valores de intensidade dos píxeles ou voxeles das imaxes a aliñar.
  }
}

\newglossaryentry{FBR}{
  name=FBR,
  description={Siglas en inglés de \textit{Feature-Based Registration}.
  Técnica de rexistro de imaxes baseada na identificación e correspondencia de características salientables, como puntos, liñas ou bordes, presentes nas imaxes.
  }
}

