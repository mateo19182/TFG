%%%%%%%%%%%%%%%%%%%%%%%%%%%%%%%%%%%%%%%%%%%%%%%%%%%%%%%%%%%%%%%%%%%%%%%%%%%%%%%%
% Obxectivo: Lista de termos empregados no documento,                          %
%            xunto cos seus respectivos significados.                          %
%%%%%%%%%%%%%%%%%%%%%%%%%%%%%%%%%%%%%%%%%%%%%%%%%%%%%%%%%%%%%%%%%%%%%%%%%%%%%%%%

\newglossaryentry{DFV}{
  name=DFV,
  description={Siglas en inglés de \textit{Campo de Vectores de Deformación}.}
}


\newglossaryentry{INR}{
  name=DFV,
  description={Siglas en inglés de \textit{Redes Neuronais Implícitas}.}
}

\newglossaryentry{MLP}{
  name=MLP,
  description={Siglas en inglés de \textit{Multi-Layer Perceptrón}.
  Tipo de red neuronal artificial formada por capas completamente conectadas de neuronas, que incluye una capa de entrada, una o más capas ocultas, y una capa de salida.
  }
}

\newglossaryentry{CNN}{
  name=CNN,
  description={Siglas en inglés de \textit{Convolutional Neural Network}.
  Tipo de red .
  }
}

\newglossaryentry{GAN}{
  name=CNN,
  description={Siglas en inglés de \textit{Generative Adversarial Network}.
  Tipo de red .
  }
}

