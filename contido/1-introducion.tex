\chapter{Introdución}
\label{chap:introducion}

\lettrine{N} este primer capítulo expóñense as motivacións e obxectivos deste traballo. Ademais, detallarase a estrutura da memoria e os apartados que a conforman.

\section{Motivación}
\label{sec:motivacion}

A oftalmoloxía válese da análise de imaxes obtidas por diversos métodos para realizar diagnósticos e seguimentos precisos.
Non obstante, dado que estas imaxes poden prover de distintas modalidades e ser tomadas dende distintos puntos no espazo ou en instantes separados no tempo, é preciso aliñalas para poder comparalas de xeito efectivo.
O aliñamento, tamén denominado rexistro, consiste en deformar dúas ou máis imaxes co obxetivo de que as características de interese se atopen na mesma posición (superpostas).
Este é un proceso tedioso e propenso a erros cando se realiza manualmente, polo que calquer mellora nel é de gran utilidade para que os profesionais da saúde, que poderán adicar máis tempo a tarefas máis relevantes.
Este problema é axeitadao para ser automatizado, xa que non require dun xuízo clínico, senón que se basea na comparación de características visuais das imaxes.

Existen diversas técnicas para realizar aliñamento de imaxes automático, especialmente ca chegada do aprendizaxe profundo á visión por computador para imaxes médicas, onde é habitual o uso de redes neuronais convolucionais (\gls{CNN}) \cite{medicalimageanalysis}.
Estos métodos, aínda que efectivos, teñen limitacións. Unha delas é que requiren dunha gran cantidade de datos para o seu adestramento, o que pode ser un problema en campos como a medicina, onde a obtención de datos é cara e complexa. Ademais, xeralmente teñen unha precisión menor ca métodos automáticos convencionais, pese a que si que son máis rápidos \cite{bharati2022deeplearningmedicalimage}.
Adaptando o traballo realizado por IDIR \cite{wolterink2021implicit}, preténdese aplicar redes de representación implícita para o aliñamento de imaxes oftalmolóxicas, para determinar se poden superar as limitacións dos métodos anteriores.

\section{Obxectivos}
\label{sec:obxectivos}

Neste traballo explorarase o uso de redes de representación implícita para o aliñamento de imaxes oftalmolóxicas, para determinar se son aptas para esta tarefa e se poden superar as limitacións dos métodos anteriores.
Para iso, os obxectivos específicos son:
\begin{itemize}
    \item Adaptar o traballo de IDIR \cite{wolterink2021implicit} para aplicalo a imaxes oftalmolóxicas de dúas dimensións.
    \item Comparar o rendemento do método proposto nos conxuntos de datos de FIRE \cite{FIRE} e RFMID \cite{RFMiD}.
    \item Analizar a influencia de distintos parámetros no rendemento, en particular a influencia da función de activación SIREN.
\end{itemize}

\section{Estrutura }
\label{sec:estrutura}

Nesta sección detallarase a estrutura da memoria e os apartados que a conforman.

\begin{itemize}
    \item \textbf{Capítulo 1: Introdución}: neste capítulo introdúcese o traballo, explicando as motivacións e obxectivos do mesmo.
    \item \textbf{Capítulo 2: Contexto}: neste capítulo explicarase o contexto do traballo, introducindo conceptos básicos de visión por computador e imaxes médicas, así como o estado da arte en aliñamento de imaxes.
    \item \textbf{Capítulo 3: Metodoloxía e planificación}: neste capítulo explicarase a metodoloxía empregada e a planificación do traballo.
    \item \textbf{Capítulo 4: Traballo realizado}: neste capítulo descríbese o traballo realizado.
    \item \textbf{Capítulo 5: Experimentos e resultados}: neste capítulo presentaranse os experimentos realizados e discutiránse os resultados obtidos. 
    \item \textbf{Capítulo 6: Conclusións}: neste capítulo resúmense as conclusións do traballo e as súas implicacións.
    \item \textbf{Capítulo 7: Traballo futuro}: neste capítulo propoñeranse liñas de traballo futuro.
\end{itemize}
