\chapter{Introdución}
\label{chap:introducion}

\lettrine{N} este primeiro capítulo expóñense as motivacións e obxectivos deste traballo. Ademais, detallarase a estrutura da memoria e os apartados que a conforman.

\section{Motivación}
\label{sec:motivacion}

A oftalmoloxía válese da análise de imaxes obtidas por diversos métodos para realizar diagnósticos e seguimentos precisos. Non obstante, dado que estas imaxes poden prover de distintas modalidades e ser tomadas dende distintos puntos no espazo ou en instantes separados no tempo, é preciso aliñalas para poder comparalas de xeito efectivo. O aliñamento, tamén denominado rexistro, consiste en deformar dúas ou máis imaxes co obxectivo de que as características de interese se atopen na mesma posición (superpostas). Este é un proceso tedioso e propenso a erros cando se realiza manualmente, polo que calquera mellora nel é de gran utilidade para os profesionais da saúde. Este problema é axeitado para ser automatizado, xa que non require dun xuízo clínico, senón que se basea na comparación de características visuais das imaxes.

Existen diversas técnicas para realizar o aliñamento automático, especialmente coa chegada da aprendizaxe profunda á visión por computador, onde é habitual o uso de redes neuronais convolucionais (CNN). Con todo, estes modelos, aínda que efectivos, presentan limitacións significativas: requiren grandes conxuntos de datos para o seu adestramento, un recurso escaso e custoso no eido médico, e xeralmente mostran unha precisión menor que os métodos convencionais.

As representacións neuronais implícitas (INRs) emerxen como un paradigma alternativo que modela a deformación como unha función continua definida nos propios pesos da rede. Esta técnica ofrece vantaxes clave, como a independencia da resolución e a capacidade de adestrarse para cada par de imaxes, eliminando a necesidade de bases de datos. A pesar do seu potencial, a súa aplicación específica ao desafío do rexistro de retinografías permanece inexplorada, presentando unha clara oportunidade de investigación.

Para abordar esta lagoa, este traballo adapta o framework IDIR, proposto por Wolterink et al. \cite{wolterink2021implicit} no eido do rexistro de pulmóns, á tarefa de aliñamento de imaxes oftalmolóxicas. O obxectivo é determinar se esta metodoloxía pode superar as limitacións dos enfoques anteriores e ofrecer unha solución robusta e precisa neste dominio.

\section{Obxectivos}
\label{sec:obxectivos}

Neste traballo explorarase o uso de redes de representación implícita para o aliñamento de imaxes oftalmolóxicas, para determinar se son aptas para esta tarefa e se poden superar as limitacións dos métodos anteriores.
Para iso, os obxectivos específicos son:
\begin{itemize}
    \item Adaptar o traballo de IDIR \cite{wolterink2021implicit} para aplicalo a imaxes oftalmolóxicas de dúas dimensións.
    \item Comparar o rendemento do método proposto nos conxuntos de datos de FIRE \cite{FIRE} e RFMID \cite{RFMiD}.
    \item Analizar a influencia de distintos parámetros no rendemento, en particular a influencia da función de activación SIREN.
\end{itemize}

\section{Estrutura }
\label{sec:estrutura}

Nesta sección detallarase a estrutura da memoria e os apartados que a conforman.

\begin{itemize}
    \item \textbf{Capítulo 1: Introdución}: neste capítulo introdúcese o traballo, explicando as motivacións e obxectivos do mesmo.
    \item \textbf{Capítulo 2: Contexto}: neste capítulo explicarase o contexto do traballo, introducindo conceptos básicos de visión por computador e imaxes médicas, así como o estado da arte en aliñamento de imaxes.
    \item \textbf{Capítulo 3: Metodoloxía e planificación}: neste capítulo explicarase a metodoloxía empregada e a planificación do traballo.
    \item \textbf{Capítulo 4: Traballo realizado}: neste capítulo descríbese o traballo realizado.
    \item \textbf{Capítulo 5: Experimentos e resultados}: neste capítulo presentaranse os experimentos realizados e discutiránse os resultados obtidos. 
    \item \textbf{Capítulo 6: Conclusións}: neste capítulo resúmense as conclusións do traballo e as súas implicacións.
    \item \textbf{Capítulo 7: Traballo futuro}: neste capítulo propoñeranse liñas de traballo futuro.
\end{itemize}
