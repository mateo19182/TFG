\chapter{Introdución}
\label{chap:introducion}

\lettrine{N} este primer capítulo expóñense as motivacións e obxetivos deste traballo. Ademais, detallarase a estrutura da memoria e os apartados que a conforman.

\section{Motivación}
\label{sec:motivacion}

A oftalmoloxía válese da análise de imaxes obtidas por diversos métodos para realizar diagnósticos e seguimentos precisos.
Non obstante, dado que estas imaxes poden prover de distintas modalidades e foron tomadas dende distintos puntos ou en instantes separados no tempo, é preciso aliñalas para poder comparalas de xeito efectivo.
O aliñamento de imaxes é un proceso que se leva a cabo para poder comparar imaxes dun mesmo paciente tomadas en distintos momentos, ou para comparar imaxes de diferentes pacientes.
Consiste en deformar dúas ou máis imaxes de forma que as características de interese se atopen na mesma posición (superpostas).
Este é un proceso tedioso e propenso a erros, polo que calquer mellora nel é de gran interese para os profesionais da saúde. 
Esta tarefa é axeitada para ser automatizada, xa que é non require dun xuízo clínico, senón que se basea na comparación de características visuais das imaxes. A automatización deste proceso permitiría aos profesionais da saúde dedicar máis tempo a tarefas máis relevantes.

Xa existen diversas técnicas para realizar aliñamento de imaxes automático, especialmente ca chegada do deep learning á visión por computador para imaxes médicas, onde unha aproximación habitual é o uso de redes neuronais convolucionais \cite{medicalimageanalysis}.
Estos métodos, aínda que efectivos, teñen limitacións. Unha delas é que requiren dunha gran cantidade de datos para o seu adestramento, o que pode ser un problema en campos como a medicina, onde a obtención de datos é cara e complexa. Ademais, xeralmente teñen unha precisión menor ca métodos automáticos convencionais, pese a que si que son máis rápidos \cite{bharati2022deeplearningmedicalimage}.
Adaptando o traballo realizado por \cite{wolterink2021implicit}, preténdese aplicar redes de representación implícita para o aliñamento de imaxes oftalmolóxicas.

\section{Obxectivos}

Neste traballo explorarase o uso de redes de representación implícita para o aliñamento de imaxes oftalmolóxicas, para determinar se son aptas para esta tarefa e se poden superar as limitacións dos métodos anteriores.
Para iso, os obxectivos específicos son:
\begin{itemize}
    \item Adaptar o traballo de IDIR \cite{wolterink2021implicit} para aplicalo a imaxes oftalmolóxicas.
    \item Comparar o rendemento do método proposto co de métodos automáticos convencionais nos datasets de FIRE \cite{FIRE} e RFMID \cite{RFMiD}.
    \item Analizar a influencia de distintos parámetros no rendemento do método proposto, en particular a influencia da función de activación SIREN.
\end{itemize}

\section{Estructura }

Nesta sección detallarase a estrutura da memoria e os apartados que a conforman.

\begin{itemize}
    \item \textbf{Capítulo 1: Introdución}: neste capítulo introdúcese o traballo, explicando as motivacións e obxetivos do mesmo.
    \item \textbf{Capítulo 2: Contexto}: neste capítulo explicarase o contexto do traballo, introducindo conceptos básicos de visión por computador e imaxes médicas, así como o estado da arte en aliñamento de imaxes.
    \item \textbf{Capítulo 3: Metodoloxía e planificación}: neste capítulo explicarase a metodoloxía empregada e a planificación do traballo.
    \item \textbf{Capítulo 4: Traballo realizado}: neste capítulo discutiranse os resultados e compararanse cos resultados de outros métodos.
    \item \textbf{Capítulo 5: Experimentos e resultados}: neste capítulo presentaranse os experimentos realizados e os resultados obtidos. 
    \item \textbf{Capítulo 6: Conclusións}: neste capítulo presentaranse as conclusións do traballo e as súas implicacións.
    \item \textbf{Capítulo 7: Traballo futuro}: neste capítulo propoñeranse liñas de traballo futuro.
\end{itemize}
