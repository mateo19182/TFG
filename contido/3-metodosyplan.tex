\chapter{Metodoloxía e planificación}
\label{chap:Metodoloxía e planificación}
\lettrine{N}{esta} sección explícase a metodoloxía de traballo empregada para o desenvolvemento do proxecto, así como a planificación do mesmo.
 Ademais, descríbense os recursos utilizados e faise unha estimación dos custos asociados ao proxecto.

\section{Metodoloxía do desenrrolo}
\label{sec:Metodoloxía do desenrrolo}

Ao ser un proxecto de investigación, a metodoloxía de traballo mais adecuada é unha metodoloxía iterativa e incremental, que permite poder adaptarse aos cambios que van xurdindo durante o desenrrolo do proxecto.

\section{Planificación do proxecto}
\label{sec:Planificación do proxecto}

Comezaráse por replicar os resultados do traballo orixinal de IDIR.

Posteriormente, implementaránse as adaptacións necesarias para o seu uso en imaxes oftalmolóxicas. Os principais cambios están principalmente relacionados coas dimensións das imaxes, xa que as imaxes de pulmóns son 4D (3 espazo + 1 tempo) e as imaxes oftalmolóxicas son 2D.

Logo disto, haberá que crear unha avaliación adecuada para o novo dominio. 

Finalmente, realizarase múltiple experimentos, que tentarán mellorar o rendemento da rede e avaliar a súa validez en imaxes oftalmolóxicas.

Tras isto, redactarase a memoria do traballo ca información obtida durante o desenrrolo do proxecto.

\section{Recursos utilizados}
\label{sec:Recursos utilizados}

\subsection{Software}
\label{subsec:Software}

Xa que parte do traballo consiste en adaptar un traballo previo, 
decidíuse empregar moito do mesmo software ca o traballo orixinal para facilitar a implementación e reproducibilidade.
O mais relevante é PyTorch, unha librería de código aberto para Python que facilita o desenrrolo de redes neuronais. Utilizáronse as versións de Python 3.12.3 e CUDA 12.2. Tamén se empregan librerías de apoio como NumPy (para traballar con matrices), Matplotlib (visualización), OpenCV ou scikit-learn (manexo de imaxes).

Outro software empregado inclúe VSCode (IDE), Git (control de versións) e LaTeX (redacción de memoria).

\subsection{Hardware}
\label{subsec:Hardware}

O proxecto foi desenrrolado nun ordenador portátil conectado por ssh a un servidor con GPU. 
Utilizáronse dous sevidores diferentes, un montado por min\footnote{\url{https://blog.m19182.dev/writings/Building-my-Homelab}} e outro facilitado polo grupo de investigación VARPA (Visión Artificial y Reconocimiento de Patrones).

A gran parte dos experimentos foron realizado no primeiro, mais para poder executar o proxecto cas imáxenes na súa resolución orixinal foi necesario empregar o segundo 
debido ás limitacións de memoria da GPU. 

\begin{table}[h]
\centering
\begin{tabular}{|c|c|c|}
\hline
\textbf{Característica} & \textbf{Homelab} & \textbf{Servidor VARPA} \\ \hline
Procesador & AMD Ryzen 9 5950X&  AMD Ryzen Threadripper 3960X \\ \hline
GPU & NVIDIA 3090 & NVIDIA RTX A6000  \\ \hline
\end{tabular}
\caption{Comparativa entre os servidores utilizados}
\label{tab:comparativa_servidores}
\end{table}


\subsection{Estimación de custos}
\label{subsec:Estimación de custos}

Os costos do hardware son ignorados xa que xa estaba disponible antes da realización do proxecto.
Os costos dos recursos humanos calcúlanse para un estudante e doús tutores, resultando nun coste estimado de $PLACEHOLDER$.

\begin{table}[h]
\centering
\begin{tabular}{|c|c|c|c|}
\hline
\textbf{Recurso} & \textbf{Coste por hora} &\textbf{Horas estimadas} & \textbf{Coste total} \\ \hline
Estudante & & & € \\ \hline
Titor 1 & & & € \\ \hline
Titor 2 & & & € \\ \hline
\end{tabular}
\caption{Estimación de custos dos recursos humanos}
\label{tab:estimacion_custos}
\end{table}