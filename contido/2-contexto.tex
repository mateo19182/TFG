\chapter{Contexto}
\label{chap:contexto}

\lettrine{N}{este} apartado introdúcense o contexto relevante a este traballo que provee os conceptos básicos necesarios para a súa comprensión.
Para elo descríbese o campo da oftalmoloxía e a imaxe médica, así como o estado da arte en aliñamento de imaxes.
\section{Oftalmoloxía}
\label{sec:Oftalmoloxía}
A oftalmoloxía é a especialidade médica que se encarga do estudo e tratamento do ollo e os seus trastornos.
 O ollo humano é un dos órganos dos que mais dependemos e maior cantidade de información sensorial aporta, e en consecuencia tamén é un dos mais complexos do noso corpo.
 Así mesmo, é unha das rexións que máis datos aporta sobre o estado de saúde do paciente, xa que permite observar directamente os vasos sanguíneos e o tecido neuronal "in-vivo".
 Isto permite a detección temprana de enfermidades, que poden ser diagnosticadas mediante a observación da retina.
\subsection{Anatomía do ollo humano}
\label{subsec:Anatomía do ollo humano}
O ollo encargase de captar a luz e transformala en impulsos eléctricos que se envían ao cerebro.
 Esta información é interpretada polo cerebro, que mediante mecanismos como a atención e a memoria, permite a percepción visual.
 \dots

\subsection{Imaxe oftalmolóxica}
\label{subsec:Imaxe oftalmolóxica}
Existen diversas modalidades de imaxe médica que permiten observar o ollo, cada unha con diferentes propiedades e aplicacións. 
As mais utilizadas son a fotografía de fondo de ollo (retinografía), a tomografía de coherencia óptica (OCT) e a angiografía con fluoresceína.

\subsection{Retinografía}
\label{subsec:Retinografía}
Este traballo céntrase na retinografía xa que é a mais común.
Isto é débese en gran parte á súa accesibilidade, requerindo equipo maís barato e menor entrenamento comparada cas outras modalidades
Ademais, é unha técnica non invasiva e rápida de realizar, o que a fai preferible na maioría dos casos.

Para realizala utilízase unha cámara especial denominada retinógrafo, e xeralmente require da previa dilatación da pupila do paciente.
Desta forma permítese maior entrada de luz nos ollos, o que provoca unha mellor visualización da retina e mellora a calidade da imaxe.
Un especialista pode analizar a retinografía para detectar signos de enfermidades como a retinopatía diabética, a hipertensión ou a degeneración macular.

imagenFIRE

\section{Rexistro de imaxes}
\label{sec:Rexistro de imaxes}
O rexistro de imaxes é un proceso que consiste en, sobre dúas ou mais imaxes, determinar a correspondencia espacial entre elas e alinealas nun sistema de coordenadas común.
Así conséguese que as características de interese se atopen na mesma posición.
Este proceso pode empregarse para comparar imaxes dun mesmo paciente tomadas en distintos momentos, en distintas modelidades ou para comparar entre diferentes pacientes.
Isto permite a revisión do avance dunha enfermidade ao longo do tempo, a fusión de imaxes de distintas modalidades ou a detección de patróns comúns en distintos individuos.

No caso de trabllar con dúas imaxes, a imaxe de referencia denomínase imaxe fixa (f) e a imaxe que se quere rexistrar imaxe móbil(m).

\subsection{Métodos de aliñamento de imaxes automáticos}
\label{subsec:Métodos de aliñamento de imaxes automáticos}
Existen diversos métodos para realizar aliñamento de imaxes, que poden estar automatizados en maior ou menor medida.
Os métodos manuais requiren da intervención dun experto para realizar o aliñamento, o que os fai pouco prácticos para grandes volumes de imaxes.

Dependendo do tipo de transformación esta pode ser clasificada en ríxida, afín ou deformable.

A ríxida tan só permite rotación e traslación, mentres que a afín permite ademais escalado e cizallamento.
Ámbas transformacións poden ser representadas por unha matriz de 2 dimensións xa que son deformacións lineais.
Ao contrario, a transformación deformable é non lineal, polo que require dunha dimensión adicional ás da imaxe a rexistrar (unha imaxe de 2d require unha matriz 3d).
Esta matriz denomínase campo de deformación, e permite representar deformacións locais na imaxe, facendoa moito mais flexible para representar transformacións complexas.

Tradicionalmente impregáronse métodos iterativos \dots

Ca chegada dos métodos de aprendizaxe profunda na imaxe médica, comezaron a empregarse redes neuronais para realizar o aliñamento de imaxes.
Estos métodos tenden a ser mais rápidos que os métodos convencionais, a custo de algo de precisión. 
\dots

\subsection{Estado da arte}
\label{subsec:Estado da arte}

\dots

\subsection{IDIR}
\label{subsec:IDIR}
IDIR (Implicit Deformable Image Registration) é un método de aliñamento de imaxes baseado en redes neuronais. 
A súa principal diferenza frente a unha rede convolucional tradicional é que, 
en lugar de predicir a transformación entre imaxes, optimízase unha rede para que represente esta transformación.


\subsubsection{Arquitectura}
\label{subsubsec:Arquitectura}

\subsubsection{Termos de regularización}
\label{subsubsec:Termos de regularización}
 based on standard automatic differentiation techniques
 
\subsubsection{Función de activación}
\label{subsubsec:Función de activación}
