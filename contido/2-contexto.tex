\chapter{Contexto}
\label{chap:contexto}

\lettrine{N}{este} apartado introdúcense o contexto relevante a este traballo que provee os conceptos básicos necesarios para a súa comprensión.
Para elo descríbese o campo da oftalmoloxía e a imaxe médica, así como o estado da arte en aliñamento de imaxes.
\section{Oftalmoloxía}
\label{sec:Oftalmoloxía}
A oftalmoloxía é a especialidade médica que se encarga do estudo e tratamento do ollo e os seus trastornos.
 O ollo humano é un dos órganos dos que mais dependemos e maior cantidade de información sensorial aporta, e en consecuencia tamén é un dos mais complexos do noso corpo.
 Así mesmo, é unha das rexións que máis datos aporta sobre o estado de saúde do paciente, xa que permite observar directamente os vasos sanguíneos e o tecido neuronal "in-vivo".
 Isto permite a detección temprana de enfermidades, que poden ser diagnosticadas mediante a observación da retina.
\subsection{Anatomía do ollo humano}
\label{subsec:Anatomía do ollo humano}
O ollo encargase de captar a luz e transformala en impulsos eléctricos que se envían ao cerebro.
 Esta información é interpretada polo cerebro, que mediante mecanismos como a atención e a memoria, permite a percepción visual.
 \dots

\subsection{Imaxe médica}
\label{subsec:Imaxe médica}
\dots
diabetes e hipertensión, así como a detección de tumores cerebrais.

\section{Aliñamento de imaxes}
\label{sec:Aliñamento de imaxes}
O aliñamento de imaxes é un proceso que consiste en deformar dúas ou máis imaxes de forma que as características de interese se atopen na mesma posición.
Este proceso é de gran importancia en campos como a medicina, onde se emprega para comparar imaxes dun mesmo paciente tomadas en distintos momentos, ou para comparar imaxes de diferentes pacientes.
Isto permite a revisión do avance dunha enfermidade ao longo do tempo, ou a comparación de imaxes de diferentes pacientes para detectar patróns comúns.

O aliñamento de imaxes é un proceso tedioso e propenso a erros, polo que calquer mellora nel é de gran interese para os profesionais da saúde, 
xa que permitiría aos profesionais da saúde dedicar máis tempo a tarefas máis relevantes.

\subsection{Métodos de aliñamento de imaxes}
\label{subsec:Métodos de aliñamento de imaxes}
Existen diversos métodos para realizar aliñamento de imaxes, que se poden clasificar en dúas categorías: métodos manuais e métodos automáticos.
Os métodos manuais requiren da intervención dun experto para realizar o aliñamento, o que os fai pouco prácticos para grandes volumes de imaxes.
Os métodos automáticos, pola súa banda, permiten realizar o aliñamento sen intervención humana, dentro destes hay diferentes enfoques.
\dots

\subsection{Estado da arte}
\label{subsec:Estado da arte}
