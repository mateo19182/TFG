\chapter{Traballo futuro}
\label{chap:Traballo futuro}

\lettrine{E}{xisten} varias liñas de traballo futuro que se poden seguir para mellorar o sistema.
A continuación, descríbense algunhas das que teñen maior potencial:

\section{Instant Neural Graphics Primitives}
\label{sec:Instant Neural Graphics Primitives}

Introducidas por \cite{mueller2022instant}, propoñen encodear os inputs da rede a un espacio dimensional superior.

Encodear os inputs da rede é unha técnica que xa se emprega en moitas ocasións (one-hot encodings, transformers...)
Eles utilizan 'sparse parametric encodings' utilizando unha tabla de hashes de múltiples resolucións, que tamén tén parámetros entrenables e fai parte do traballo de aprendizaxe da rede.
Isto permítelles un entrenamento e inferencia moito mais rápido que outros métodos, sen ter que sacrificar en rendemento.

\cite{li2024neuralgraphicsprimitivesdeformable} aplicao estas ideas á tarefa de rexistro, con moi bós resultados.
Notablemente, resuelven o 'sliding boundary problem', que se refiere ás complicación de modelar o movemento relativo entre diferentes estructuras. 
No caso da imaxe pulmonar, surxe cuando os lóbulos dos pulmóns se deslizan entre sí durante la respiración.



\section{Invertibilidade}
\label{sec:Invertibilidade}

Os resultados de esta rede non teñen ningunha garantía de ser difeomórficos, 
mais debido aos termos de regularización utilizados durante o adestramento son poucos os casos nos que o determinante jacobiano é negativo (o que indicaría que a transformación non é invertible).

É posible garantir a invertibilidade da transfomación facendo uso de redes invertibles \cite{jacobsen2018irevnetdeepinvertiblenetworks}.

Tamén con campos de velocidade neuronais \cite{sun2024medicalimageregistrationneural} ...