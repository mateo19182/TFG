\chapter{Traballo futuro}
\label{chap:Traballo futuro}

\lettrine{E}{xisten} varias liñas de traballo futuro que se poden seguir para mellorar o sistema actual.
 Os resultados obtidos neste traballo, aínda que demostran a viabilidade de adaptar o framework IDIR para o aliñamento de imaxes oftalmolóxicas 2D, tamén revelan limitacións á hora de acadar a precisión e robustez desexables.
 As seguintes liñas de traballo futuro considéranse prometedoras para superar estes desafíos e avanzar no campo:

% \section{Instant Neural Graphics Primitives}
% \label{sec:Instant Neural Graphics Primitives}

% Introducidas por \cite{mueller2022instant}, propoñen encodear os inputs da rede a un espacio dimensional superior.

% Encodear os inputs da rede é unha técnica que xa se emprega en moitas ocasións (one-hot encodings, transformers...)
% Eles utilizan 'sparse parametric encodings' utilizando unha tabla de hashes de múltiples resolucións, que tamén tén parámetros entrenables e fai parte do traballo de aprendizaxe da rede.
% Isto permítelles un adestramento e inferencia moito mais rápido que outros métodos, sen ter que sacrificar en rendemento.

% \cite{li2024neuralgraphicsprimitivesdeformable} aplicao estas ideas á tarefa de rexistro, con moi bós resultados.
% Notablemente, resuelven o 'sliding boundary problem', que se refiere ás complicación de modelar o movemento relativo entre diferentes estructuras. 
% No caso da imaxe pulmonar, surxe cuando os lóbulos dos pulmóns se deslizan entre sí durante la respiración.

\section{Arquitecturas alternativas}
\label{sec:Arquitecturas alternativas}

Unha liña relevante de traballo futuro é a exploración de arquitecturas alternativas. 
Mentres que os perceptróns multicapa (MLPs) son considerados aproximadores universais \cite{HORNIK1989359} (son capaces de aproximar calquera función continua dada unha cantidade suficiente de neuronas), é posible que a arquitectura utilizada de 3 capas con 256 neuronas por capa non sexa o suficientemente grande para capturar as complexidades das transformacións entre as retinografías.

Unha opción sería aumentar o número de capas ou neuronas por capa. Outra sería implementar o uso de codificación posicional, que parece ser útil para a tarefa de rexistro \cite{mueller2022instant}.

Outra idea moi intersante é o uso de restriccións de consistencia cíclicas, propostas por Van Harten et al. no contexto de rexistro de imaxes médicas \cite{van_Harten_2024}. Consiste en entrenar dúas redes á vez que estiman as transformacións directas e inversas (unha da fixa á móbil e outra da móbil á fixa), facendo que estas se regularicen mutuamente e estabilizando a optimización.
Unha vantaxe deste enfoque é que produe unha métrica de certidumbre ao comparar as transformacións estimadas, o cal é moi útil en aplicacións clínicas.

Tamén podía ser interesante explorar o uso de meta-aprendizaxe, onde se aprende unnha inicialización de pesos óptima a partir dun conxunto de datos \cite{learnedinit}, ou condicionar por xeometría, onde se incorpora coñecemento anatómico previo para simplificar a complexidade da deformación \cite{harten2023deformable}.  

\section{Invertibilidade}
\label{sec:Invertibilidade}

Unha direción interesante para o traballo futuro é a exploración de métodos que garantan a invertibilidade das transformacións aprendidas pola rede.
A rede IDIR actual non garante a invertibilidade das transformacións aprendidas, o que significa que non é posíbel aplicar a transformación inversa de maneira fiable.

Grazas aos termos de regularización utilizados durante o adestramento son poucos os casos nos que o determinante Xacobiano é negativo (o que indicaría que a transformación non é invertible).

Aproximación como a de i-RevNet \cite{jacobsen2018irevnetdeepinvertiblenetworks} ou aqueles baseados en campos vectoriais de velocidade \cite{sun2024medicalimageregistrationneural} permiten garantir a invertibilidade das transformacións aprendidas, o que podería mellorar a precisión e a robustez do rexistro e funcionaría como un mecanismo de regulación implícita.

\section{Enfoque híbrido}
\label{sec:Enfoque híbrido}

Outra liña de traballo futuro é a exploración de enfoques híbridos que combinen o rexistro baseado en redes neuronais con técnicas tradicionais de rexistro.
Unha posibilidade sería utilizar o rexistro tradicional para proporcionar un rexistro inicial e global robusto, que despois podería ser refinado por unha rede neuronal.

Máis concretamente, consistiría en utilizar un detector de puntos clave robusto como SuperPoint \cite{superpoint} para extraer características das imaxes fixa e móbil, e utilizar un algoritmo de emparellamento de puntos clave como SuperGlue \cite{superglue} para obter unha transformación inicial entre as imaxes.
Posteriormente adestraría o modelo INR para refinar esta transformación inicial, o que é un problema de optimización máis sinxelo e que fai uso das vantaxes de ambos os enfoques.

Este é o enfoque utilizado por métodos no estado da arte como HybridRetina \cite{liu2024progressiveretinalimageregistration}. 

Así mesmo, poderíanse explorar con máis profundidade o preprocesado das imaxes, xa que é inexistente no método actual pero podería ser útil para mellorar a calidade das imaxes e facilitar o rexistro.