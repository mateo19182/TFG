\chapter{Contido demostrativo}
\label{chap:demo}

\lettrine{E}{ntre} a introdución e as conclusións, o documento conterá
tantos capítulos como sexa preciso, sempre con coidado de non rebasar
o límite de 80 páxinas fixado polo regulamento de TFGs.

Empregaremos éste de xeito demostrativo, para ilustrar o uso de
elementos habituais que poidan ser de utilidade\footnote{Por exemplo,
  isto é unha nota a pé de páxina.}.

\section{Inclusión de imaxes}

Se precisamos imaxes no noso documento, incluirémolas do xeito que se
indica na figura~\ref{fig:exemplo} (páxina~\pageref{fig:exemplo}). Se
o facemos así, \LaTeX ubicará cada imaxe no mellor lugar posible,
lugar que pode variar a medida que o documento vaia crecendo coa
inclusión de máis texto e outros elementos (máis imaxes, táboas,
etc.).

\begin{figure}[hp!]
  \centering
  \includegraphics[width=0.75\textwidth]{imaxes/udc.png}
  \caption{Pé de imaxe descritivo}
  \label{fig:exemplo}
\end{figure}

Recoméndase almacenar os ficheiros gráficos no directorio
\texttt{imaxes}.

\subsection{Inclusión de varias sub-imaxes}

Se precisamos inserir imaxes relacionadas, pode ser apropiado
incluílas como sub-figuras, do xeito que se pode apreciar na
figura~\ref{fig:exemplo-subfiguras} (páxina~\pageref{fig:exemplo-subfiguras})
coas imaxes~\ref{fig:subfigura-rotada} e~\ref{fig:subfigura-deformada}.
Como se pode ver nos exemplos desta sección, sempre é recomendable
referirse ás imaxes (ou táboas e outros elementos \emph{flotantes},
que se demostrarán nas seccións seguintes deste capítulo demostrativo)
pola súa referencia, xa que dese xeito non dependemos de onde
queden ubicados os elementos en cuestión.

\begin{figure}[hp!]
  \centering
  \begin{subfigure}[c]{0.3\textwidth}
    \includegraphics[angle=45,width=\textwidth]{imaxes/udc.png}
    \caption{Pé de subimaxe rotada}
    \label{fig:subfigura-rotada}
  \end{subfigure}
  \hspace{0.1\textwidth}
  \begin{subfigure}[c]{0.3\textwidth}
    \includegraphics[width=\textwidth,height=3cm]{imaxes/udc.png}
    \caption{Pé de subimaxe deformada}
    \label{fig:subfigura-deformada}
  \end{subfigure}
  \caption{Pé de imaxe xeral}
  \label{fig:exemplo-subfiguras}
\end{figure}

\section{Inclusión de táboas}

Se precisamos táboas no noso documento, incluirémolas do xeito que se
indica na táboa~\ref{tab:exemplo} (páxina~\pageref{tab:exemplo}). Se
o facemos así, \LaTeX ubicará cada táboa no mellor lugar posible,
lugar que pode variar a medida que o documento vaia crecendo coa
inclusión de máis texto e outros elementos (máis imaxes, táboas,
etc.).

\begin{table}[hp!]
  \centering
  \rowcolors{2}{white}{udcgray!25}
  \begin{tabular}{c|c}
  \rowcolor{udcpink!25}
  \textbf{Título de columna} & \textbf{Outro título de columna} \\\hline
  \textit{Título de fila} & Contido da cela \\
  \textit{Título de fila} & Contido da cela \\
  \textit{Título de fila} & Contido da cela \\
  \textit{Título de fila} & Contido da cela \\
  \textit{Título de fila} & Contido da cela \\
  \textit{Título de fila} & Contido da cela \\
  \end{tabular}
  \caption{Pé de táboa descritivo}
  \label{tab:exemplo}
\end{table}

\subsection{Inclusión de táboas longas}

Para táboas longas que ocupan varias páxinas, como é o caso da \ref{tab:longa}
(páxina~\pageref{tab:longa}), recoméndase o uso do paquete \texttt{lontable},
incluído xa entre os paquetes recomendados no ficheiro raíz do proxecto
(\verb+memoria_tfg.tex+).

\input{contido/taboa-longa.tex}

\subsection{Inclusión de táboas con celas que ocupan varias columnas ou filas}

En ocasións pode resultar de interese incluír nunha táboa unha cela que se estenda
a través de varias columnas, como ocorre na táboa~\ref{tab:exemplocolumnas}
(páxina~\pageref{tab:exemplocolumnas}).

\begin{table}[hp!]
  \centering
  \rowcolors{2}{white}{udcgray!25}
  \begin{tabular}{c|c|c}
  \rowcolor{udcpink!25}
  \multicolumn{3}{c}{\textbf{Cela en varias columnas}} \\\hline
  \rowcolor{udcpink!25}
  \textbf{Título de columna} & \textbf{Outro título de columna} & \textbf{Outro título máis} \\\hline
  \textit{Título de fila}    & Contido da cela                  & Contido da cela \\
  \textit{Título de fila}    & Contido da cela                  & Contido da cela \\
  \textit{Título de fila}    & \multicolumn{2}{c}{Contido da cela múltiple} \\
  \textit{Título de fila}    & Contido da cela                  & Contido da cela \\
  \end{tabular}
  \caption{Pé de táboa descritivo (táboa con celas que ocupan varias columnas)}
  \label{tab:exemplocolumnas}
\end{table}

Tamén pode resultar necesario facer o propio mais en varias filas da mesma columna,
como ocorre na táboa~\ref{tab:exemplofilas} (páxina~\pageref{tab:exemplofilas}).
Para isto é preciso o paquete \texttt{multirow}, incluído entre os recomendados no
ficheiro raíz do proxecto (\verb+memoria_tfg.tex+).

O uso de celas multifila requerirá do xuste da coloración das filas, a fin de manter
a coherencia entre o contido e o continente. Así, no canto de usar un único comando
\verb+rowcolors+ para indicar a alternancia en toda a táboa, usaremos o comando
\verb+rowcolor+ antes dunha fila que queiramos colorear, e o comando \verb+cellcolor+
dentro dunha cela que queiramos colorear.

\begin{table}[hp!]
  \centering
  \begin{tabular}{c|c}
  \rowcolor{udcpink!25}
  \textbf{Título de columna} & \textbf{Outro título de columna} \\\hline
  \multirow{2}{*}{\textit{Título de fila}} & \cellcolor{udcgray!25} Contido da cela \\
                                           & Contido da cela \\
  \rowcolor{udcgray!25}
  \textit{Título de fila}                  & Contido da cela \\
  \multirow{3}{*}{\textit{Título de fila}} & Contido da cela \\
                                           & \cellcolor{udcgray!25} Contido da cela \\
                                           & Contido da cela \\
  \rowcolor{udcgray!25}
  \textit{Título de fila}                  & Contido da cela \\
  \end{tabular}
  \caption{Pé de táboa descritivo (táboa con celas que ocupan varias filas)}
  \label{tab:exemplofilas}
\end{table}

Por suposto, pódense combinar nunha mesma táboa os dous tipos de celas (as que se 
estenden máis dunha fila e máis dunha columna), como na táboa~\ref{tab:exemplofilasecolumnas}
(páxina~\pageref{tab:exemplofilasecolumnas}).

\begin{table}[hp!]
  \centering
  \begin{tabular}{c|c|c}
  \rowcolor{udcpink!25}
  \multicolumn{3}{c}{\textbf{Cela en varias columnas}} \\\hline
  \rowcolor{udcpink!25}
  \textbf{Título de columna}               & \textbf{Outro título de columna}             & \textbf{Outro título máis} \\\hline
  \multirow{2}{*}{\textit{Título de fila}} & \cellcolor{udcgray!25} Contido da cela       & \cellcolor{udcgray!25} Contido da cela \\
                                           & Contido da cela                              & Contido da cela \\
  \rowcolor{udcgray!25}
  \textit{Título de fila}                  & \multicolumn{2}{c}{Contido da cela múltiple} \\
  \multirow{3}{*}{\textit{Título de fila}} & Contido da cela                              & Contido da cela \\
                                           & \multicolumn{2}{c}{\cellcolor{udcgray!25} Contido da cela múltiple} \\
                                           & \multicolumn{2}{c}{Contido da cela múltiple} \\
  \rowcolor{udcgray!25}
  \textit{Título de fila}                  & Contido da cela                              & Contido da cela \\
  \end{tabular}
  \caption{Pé de táboa descritivo (táboa con celas que ocupan varias columnas)}
  \label{tab:exemplofilasecolumnas}
\end{table}

\section{Inclusión de código fonte}

Se precisamos incluír fragmentos de código fonte, podemos facelo, por exemplo, da
seguinte maneira:

\begin{lstlisting}[language=C]
#include <stdio.h>
#define N 10

int main()
{
  int i;

  // Isto é un comentario
  puts("Ola, mundo!");

  for (i = 0; i < N; i++)
  {
    puts("LaTeX é a ferramenta de edición ideal para profesionais da informática!");
  }

  return 0;
}
\end{lstlisting}

\section{Uso da relación de acrónimos e do glosario}

Os acrónimos edítanse no ficheiro \texttt{bibliografia/acronimos.tex}
e úsanse empregando a orde \texttt{acrlong} para obter o termo
completo (deste xeito: \acrlong{erlang}), a orde \texttt{acrshort}
para obter o acrónimo (deste xeito: \acrshort{erlang}). A primeira vez
que usamos un termo con acrónimo no documento é recomendable usar orde
\texttt{acrfull} (que produce ambas versións á vez:
\acrfull{erlang}). Os acrónimos que non se usan no documento, non
aparecen na relación que se xerar na versión PDF.

Pola súa banda, os termos do glosario edítanse no ficheiro
\texttt{bibliografia/glo\-sa\-rio.tex} e úsanse empregando a orde
\texttt{gls} (deste xeito, \gls{bytecode}) ou \texttt{Gls} (deste
xeito, \Gls{bytecode}). Ao igual que os acrónimos, os termos que non
se usan no documento, non aparecen na relación que se xera na versión
PDF.
