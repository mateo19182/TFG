%%%%%%%%%%%%%%%%%%%%%%%%%%%%%%%%%%%%%%%%%%%%%%%%%%%%%%%%%%%%%%%%%%%%%%%%%%%%%%%%

\pagestyle{empty}
\begin{abstract}
O aliñamento da imaxe oftalmolóxica é un campo moi relevante. Aliñar imaxes médicas é útil para,
entre outras cousas, revisar o avance dunha enfermidade ao longo do tempo. O caso dos ollos é de particular importancia xa que permiten a observación in-vivo de tecido neuronal e vasos sanguíneos. Aliñar as imaxes manualmente é un proceso tedioso e complexo, polo que automatizar este proceso é moi beneficioso.

Neste traballo explórase o uso de redes de representación implícita,
 onde se parametriza a imaxe como unha función continua coas coordenadas como entrada e o valor do pixel como saída, como unha alternativa para o aliñamento de imaxes.
 Estas aportan vantaxes frente a representacións tradicionais discretas como a independencia de resolución e poder prescindir de grandes bases de datos xa que se adestran mediante un proceso de optimización para cada par de imaxes.
 Ademais, en lugar de usar funcións de activación estándar como RELU, adoitan empregar unha función de activación sinusoidal (SIREN), que pode axudar a eliminar o sesgo cara sinais de baixa frecuencia e mapear mellor deformación pequenas e detalladas.

Adaptando o traballo realizado por \cite{wolterink2021implicit}, valoraráse se este método é apto para a tarefa de aliñamento de imaxes oftalmolóxicas e como se compara con métodos convencionais.

  \vspace*{25pt}
  \begin{segundoresumo}
    Ophthalmic image alignment is a highly relevant field. Aligning medical images is useful for, among other things, reviewing the progression of a disease over time. The case of eyes is particularly important as they allow in-vivo observation of neuronal tissue and blood vessels. Manually aligning images is a tedious and complex process, so automating this process is beneficial.
    
    This work explores the use of implicit representation networks, where the image is parameterized as a continuous function with coordinates as input and pixel value as output. This provides advantages over traditional discrete representations such as resolution independence and the ability to dispense with large databases since they are trained through an optimization process for each group of images.
    Furthermore, instead of using standard activation functions like RELU, they typically employ a sinusoidal activation function (SIREN), which can help eliminate bias towards low-frequency signals and better map small and detailed deformations.
    
    Based on the work done by \cite{wolterink2021implicit}, this study will evaluate whether this method is suitable for the task of aligning ophthalmic images and how it compares to conventional methods.
  \end{segundoresumo}
\vspace*{25pt}
\begin{multicols}{2}
  \begin{description}
  \item [\palabraschaveprincipal:] \mbox{} \\[-20pt]
  \begin{itemize}
      \item Imagen médica
      \item Imagen oftalmológica
      \item Aprendizaje profundo
      \item Registro de Imágenes 
      \item Representaciones neuronales implícitas
  \end{itemize}
  
  \end{description}
  \begin{description}
  \item [\palabraschavesecundaria:] \mbox{} \\[-20pt]
  \begin{itemize}
      \item Medical imaging
      \item Ophthalmological imaging
      \item Deep learning
      \item Image Registration
      \item Implicit neural representations (INRs)
  \end{itemize}
  \end{description}
  \end{multicols}
\end{abstract}
\pagestyle{fancy}

%%%%%%%%%%%%%%%%%%%%%%%%%%%%%%%%%%%%%%%%%%%%%%%%%%%%%%%%%%%%%%%%%%%%%%%%%%%%%%%%
