%%%%%%%%%%%%%%%%%%%%%%%%%%%%%%%%%%%%%%%%%%%%%%%%%%%%%%%%%%%%%%%%%%%%%%%%%%%%%%%%

\pagestyle{empty}
\begin{abstract}
O aliñamento da imaxe oftalmolóxica é un campo moi relevante. Aliñar imaxes médicas é útil para,
entre outras cousas, revisar o avance dunha enfermidade ao longo do tempo. O caso dos ollos é de particular importancia xa que permiten a observación in-vivo de tecido neuronal e de vasos sanguíneos. Aliñar as imaxes manualmente é un proceso longo e complexo, que non se pode levar a cabo no día a día da práctica clinica. Desta forma, automatizar este proceso é moi beneficioso.

Neste traballo explórase o uso de redes de representación implícita, onde se parametriza a imaxe como unha función continua coas coordenadas como entrada e o valor do pixel como saída.. Isto aporta vantaxes frente a representacións tradicionais discretas como independencia de resolución e poder prescindir de grandes bases de datos para o adestramento xa que se adestran mediante un proceso de optimización para cada imaxe. Ademais, en lugar de usar funcións de activación estándar como RELU, poden empregar unha función de activación sinusoidal (SIREN), que pode axudar a eliminar o sesgo cara sinais de baixa frecuencia e mapear mellor deformación pequenas e detalladas.

Adaptando o traballo feito por \cite{wolterink2021implicit} aplicamolo a FIRE e RFMID, 

  \vspace*{25pt}
  \begin{segundoresumo}
    \blindtext % substitúe este comando polo resumo do teu TFG
               % na lingua secundaria do documento (tipicamente: inglés)
  \end{segundoresumo}
\vspace*{25pt}
\begin{multicols}{2}
  \begin{description}
  \item [\palabraschaveprincipal:] \mbox{} \\[-20pt]
  \begin{itemize}
      \item Imagen médica
      \item Imagen oftalmológica
      \item Palabra chave 3
      \item Palabra chave 4
      \item Palabra chave 5
  \end{itemize}
  
  \end{description}
  \begin{description}
  \item [\palabraschavesecundaria:] \mbox{} \\[-20pt]
  \begin{itemize}
      \item Medical imaging
      \item Ophthalmological imaging
      \item Palabra chave 3
      \item Palabra chave 4
      \item Palabra chave 5
  \end{itemize}
  \end{description}
  \end{multicols}
  
\end{abstract}
\pagestyle{fancy}

%%%%%%%%%%%%%%%%%%%%%%%%%%%%%%%%%%%%%%%%%%%%%%%%%%%%%%%%%%%%%%%%%%%%%%%%%%%%%%%%
