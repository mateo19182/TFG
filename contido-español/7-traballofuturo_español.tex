\chapter{Trabajo futuro}
\label{chap:Traballo futuro}

\lettrine{E}{xisten} varias líneas de trabajo futuro que se pueden seguir para mejorar el sistema actual.
Los resultados obtenidos en este trabajo, aunque demuestran la viabilidad de adaptar el framework IDIR para el alineamiento de imágenes oftalmológicas 2D, también revelan limitaciones a la hora de alcanzar la precisión y robustez deseables.
Las siguientes líneas de trabajo futuro se consideran prometedoras para superar estos desafíos y avanzar en el campo:

\section{Arquitecturas alternativas}
\label{sec:Arquitecturas alternativas}

Una línea relevante de trabajo futuro es la exploración de arquitecturas alternativas.
Mientras que los perceptrones multicapa (MLPs) son considerados aproximadores universales \cite{HORNIK1989359} (son capaces de aproximar cualquier función continua dada una cantidad suficiente de neuronas), es posible que la arquitectura utilizada de 3 capas con 256 neuronas por capa no sea lo suficientemente grande para capturar las complejidades de las transformaciones entre las retinografías.

Una opción sería aumentar el número de capas o neuronas por capa. Otra sería implementar el uso de codificación posicional, que parece ser útil para la tarea de registro \cite{mueller2022instant}.

Otra idea muy interesante es el uso de restricciones de consistencia cíclicas, propuestas por Van Harten et al. en el contexto de registro de imágenes médicas \cite{van_Harten_2024}. Consiste en entrenar dos redes a la vez que estiman las transformaciones directas e inversas (una de la fija a la móvil y otra de la móvil a la fija), haciendo que estas se regularicen mutuamente y estabilizando la optimización.
Una ventaja de este enfoque es que produce una métrica de certidumbre al comparar las transformaciones estimadas, lo cual es muy útil en aplicaciones clínicas.

También podría ser interesante explorar el uso de meta-aprendizaje, donde se aprende una inicialización de pesos óptima a partir de un conjunto de datos \cite{learnedinit}, o condicionar por geometría, donde se incorpora conocimiento anatómico previo para simplificar la complejidad de la deformación \cite{harten2023deformable}.

\section{Invertibilidad}
\label{sec:Invertibilidade}

Una dirección interesante para el trabajo futuro es la exploración de métodos que garanticen la invertibilidad de las transformaciones aprendidas por la red.
La red IDIR actual no garantiza la invertibilidad de las transformaciones aprendidas, lo que significa que no es posible aplicar la transformación inversa de manera fiable.

Gracias a los términos de regularización utilizados durante el entrenamiento son pocos los casos en los que el determinante Jacobiano es negativo (lo que indicaría que la transformación no es invertible).

Aproximaciones como la de i-RevNet \cite{jacobsen2018irevnetdeepinvertiblenetworks} o aquellos basados en campos vectoriales de velocidad \cite{sun2024medicalimageregistrationneural} permiten garantizar la invertibilidad de las transformaciones aprendidas, lo que podría mejorar la precisión y la robustez del registro y funcionaría como un mecanismo de regulación implícita.

\section{Enfoque híbrido}
\label{sec:Enfoque híbrido}

Otra línea de trabajo futuro es la exploración de enfoques híbridos que combinen el registro basado en redes neuronales con técnicas tradicionales de registro.
Una posibilidad sería utilizar el registro tradicional para proporcionar un registro inicial y global robusto, que después podría ser refinado por una red neuronal.

Más concretamente, consistiría en utilizar un detector de puntos clave robusto como SuperPoint \cite{superpoint} para extraer características de las imágenes fija y móvil, y utilizar un algoritmo de emparejamiento de puntos clave como SuperGlue \cite{superglue} para obtener una transformación inicial entre las imágenes.
Posteriormente entrenaría el modelo INR para refinar esta transformación inicial, lo que es un problema de optimización más sencillo y que hace uso de las ventajas de ambos enfoques.

Este es el enfoque utilizado por métodos en el estado del arte como HybridRetina \cite{liu2024progressiveretinalimageregistration}.

Asimismo, se podrían explorar con más profundidad el preprocesado de las imágenes, ya que es inexistente en el método actual pero podría ser útil para mejorar la calidad de las imágenes y facilitar el registro.